\documentclass{article} % For LaTex2e
\usepackage{iclr2022_conference,times}
% Optional math commands from https://github.com/goodfeli/dlbook_notation.
\input{math_commands.tex}

%######## AER210: Uncomment your submission name
\newcommand{\aername}{ - Lab Report}
%\newcommand{\aername}{Progress Report}
%\newcommand{\aername}{Final Report}

%######## AER210: Put your Group Number here
%\newcommand{\gpnumber}{40}

\usepackage{hyperref}
\usepackage{xcolor}
\usepackage[normalem]{ulem}
\usepackage{url}
\usepackage{graphicx}
\usepackage{placeins}
\usepackage{float}
\usepackage{tikz}
\usepackage{multicol}

%######## AER210: Put your project Title here
\title{Real-Time Neural Signal Filtering via \\
Hodgkin-Huxley Simulation Models}

%######## AER210: Put your names, student IDs and Emails here
\author{
    Peter Leong \\
    Student\# 1010892955 \\
    peter.leong@mail.utoronto.ca \\
\AND
}

% The \author macro works with any number of authors. There are two commands
% used to separate the names and addresses of multiple authors: \And and \AND.
%
% Using \And between authors leaves it to \LaTex{} to determine where to break
% the lines. Using \AND forces a linebreak at that point. So, if \LaTex{}
% puts 3 of 4 authors names on the first line, and the last on the second
% line, try using \AND instead of \And before the third author name.

\newcommand{\fix}{\marginpar{FIx}}
\newcommand{\new}{\marginpar{NEW}}

\iclrfinalcopy 
%######## AER210: Document starts here
\begin{document}

\maketitle

\vspace{-6ex}

\begin{abstract}
%######## AER210: Do not change the next line. This shows your Main body page count.
----Total Pages: \pageref{last_page}
\end{abstract}

\vspace{2ex}

\begin{multicols}{2}

\section{Introduction}

\section{Methods \& Procedure}

This section outlines various experiment procedure and also the preparation of equipment including Flowcoach operation and FlowEx.

\subsection{Flowcoach Apparatus}

The main purpose of the Flowcoach apparatus is to examine the flow of molecules around various model inserts.
These visualizations are obtained using air or foam bubbles, or Particle Image Velocimetry (PIV) which is described in later sections.

\subsubsection{Flowcoach Operation}

First, the main reservoir was filled till the water level was around 10cm from the bottom while the system was running.
Then, the prescribed insert model was placed into the test section, ensuring that for the venturi model, the pressure ports were aligned.
Afterward, the cap on the upper reservoir and the reservoir vent valve were securely closed.
Next, the water level in the upper reservoir was increased to fully cover the agitator.
Finally, the flow rate was controlled to the desired value using the rotameter control knob.

\subsubsection{Flow Visualization}

The flow of molecules can be physically observed by generating bubbles or foam.
To generate bubbles, the red reservoir vent valve was opened and the flow rate was increased above 1.3 GPM.
To stop generating bubbles, the red reservoir vent valve was closed.

To generate foam, the main pump was turned off and the cap on the upper reservoir was removed.
A small drop of foam solution was added into the reservoir which was then closed.


\subsection{Computational Fluid Dynamics Solver}


\section{Technical Background}

\section{Conclusion}

\label{last_page}

\newpage
\bibliographystyle{iclr2022_conference}
\bibliography{AER210_Proposal_Ref}

\end{multicols}
\end{document}